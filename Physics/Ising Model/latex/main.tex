\documentclass[a4paper, 11pt]{article}

\usepackage[a4paper, width=150mm, top=14mm, bottom=14mm, headheight=14mm]{geometry}

\usepackage{graphicx}
\usepackage{subcaption}
\usepackage{import}
\usepackage{braket}
\captionsetup{font=Large}

\begin{document}

\vspace*{\fill}
\centering \Large Results from Monte Carlo simulation of the Ising model using metropolis algorithm.
The default lattice size is $1000 \times 1000$, the default number of iterations is
$10$ billion and the default value of h is zero unless otherwise stated.
\vspace*{\fill}
\clearpage

\begin{figure}[t]
	\begin{subfigure}{\textwidth}
		\centering
		\def\svgwidth{\columnwidth}
		\import{./figures/}{convergence_iterations.pdf_tex}
		\caption{Convergence for Infinite Iterations.}
		\vspace{1.5em}
	\end{subfigure}
	\vfill
	\begin{subfigure}{\textwidth}
		\centering
		\def\svgwidth{\columnwidth}
		\import{./figures/}{convergence_lattice_size.pdf_tex}
		\caption{\centering Convergence for Infinite Lattice (fluctuations decrease for bigger
		         lattice). Iterations per Lattice Point is Kept Constant.}
		\vspace{1em}
	\end{subfigure}
	\caption{Convergence of Spontaneous Magnetization.}
\end{figure}

\begin{figure}[t]
	\begin{subfigure}{\textwidth}
		\centering
		\def\svgwidth{\columnwidth}
		\import{./figures/}{magnetization_1billion_iterations.pdf_tex}
		\caption{1 billion Iterations.}
		\vspace{1.5em}
	\end{subfigure}
	\vfill
	\begin{subfigure}{\textwidth}
		\centering
		\def\svgwidth{\columnwidth}
		\import{./figures/}{magnetization.pdf_tex}
		\caption{10 billion Iterations.}
		\vspace{1em}
	\end{subfigure}
	\caption{Convergence of Ensemble Average.}
\end{figure}

\begin{figure}[t]
	\begin{subfigure}{0.485\textwidth}
		\centering
		\def\svgwidth{\columnwidth}
		\import{./figures}{magnetic_domain_1.pdf_tex}
		\caption{$T/T_c = 0.01 \hspace{1em} \braket{s} = 1.00$}
		\vspace{1em}
	\end{subfigure}
	\hfill
	\begin{subfigure}{0.485\textwidth}
		\centering
		\def\svgwidth{\columnwidth}
		\import{./figures}{magnetic_domain_2.pdf_tex}
		\caption{$T/T_c = 0.50 \hspace{1em} \braket{s} = 1.00$}
		\vspace{1em}
	\end{subfigure}
	\vfill
	\begin{subfigure}{0.485\textwidth}
		\centering
		\def\svgwidth{\columnwidth}
		\import{./figures}{magnetic_domain_3.pdf_tex}
		\caption{$T/T_c = 0.95 \hspace{1em} \braket{s} = 0.83$}
		\vspace{1em}
	\end{subfigure}
	\hfill
	\begin{subfigure}{0.485\textwidth}
		\centering
		\def\svgwidth{\columnwidth}
		\import{./figures}{magnetic_domain_4.pdf_tex}
		\caption{$T/T_c = 1.00 \hspace{1em} \braket{s} = 0.50$}
		\vspace{1em}
	\end{subfigure}
	\vfill
	\begin{subfigure}{0.485\textwidth}
		\centering
		\def\svgwidth{\columnwidth}
		\import{./figures}{magnetic_domain_5.pdf_tex}
		\caption{$T/T_c = 1.05 \hspace{1em} \braket{s} = -0.01$}
		\vspace{1em}
	\end{subfigure}
	\hfill
	\begin{subfigure}{0.485\textwidth}
		\centering
		\def\svgwidth{\columnwidth}
		\import{./figures}{magnetic_domain_6.pdf_tex}
		\caption{$T/T_c = 1.50 \hspace{1em} \braket{s} = 0.00$}
		\vspace{1em}
	\end{subfigure}
	\caption{\centering Magnetic Domains at Different Temperatures (black refers to up spin and
	         white refers to down spin).}
\end{figure}

\begin{figure}[t]
	\begin{subfigure}{\textwidth}
		\centering
		\def\svgwidth{\columnwidth}
		\import{./figures/}{thermodynamic_functions.pdf_tex}
		\caption{Average Energy and Heat Capacity in Absence of External Magnetic Field.}
		\vspace{1.5em}
	\end{subfigure}
	\vfill
	\begin{subfigure}{\textwidth}
		\centering
		\def\svgwidth{\columnwidth}
		\import{./figures/}{thermodynamic_functions_for_different_h_norm.pdf_tex}
		\caption{\centering Average Energy for Different Values of External Magnetic Field (exact
		solution is for $h = 0$ and is almost completely overlapped by the simulated data).}
		\vspace{1em}
	\end{subfigure}
	\caption{Thermodynamic Functions.}
\end{figure}

\begin{figure}[t]
	\centering
	\def\svgwidth{\columnwidth}
	\import{./figures/}{magnetization_for_different_h_norm.pdf_tex}
	\caption{\centering Magnetization for Different Magnetic Fields (It is almost unaffected for
             $T < T_c$).}
\end{figure}

\end{document}
